\documentclass{article} 
\usepackage{amsmath} % these two packages are needed to support the typesetting of most math-related LaTeX commands/environments.
\usepackage{amssymb}

\title{Lab 8 Notes\\ {\small Typesetting MATH in \LaTeX}}
\author{Module Convenor}
\date{14/04/2025}

\begin{document}
\maketitle

%%%%%%%%%%%%%%%%%%%%%%%%%%%%%%%%%%%%%
\subsection{\textit{(review)} inline math and display math}


The mass-energy equivalence is described by the famous equation
\[ E=mc^2 \]
discovered in 1905 by Albert Einstein. In natural units ($c=1$), the formula expresses the identity
$$  E=m. $$


\noindent If $f$ is continuous on the interval $[a,b]$ and $F$ is any antiderivative of $f$, then
\[
\int _{a}^{b} f(x)\,dx =F(b)-F(a)
\]
This is known as \textit{Fundamental Theorem of Calculus}.



%%%%%%%%%%%%%%%%%%%%%%%%%%%%%%%%%%%%%
\subsection{equation environment}


\begin{equation}
\log_{27} 81=\frac{\log_3 81}{\log_3 27}=\frac{4}{3}
\end{equation}



\begin{equation}
\frac{d}{dx} (\sec x )= \sec x \cdot \tan x
\end{equation}

\begin{equation}
\sum_{n=1}^{\infty} \frac{1}{n^2}=1+\frac{1}{2^2}+\frac{1}{3^2}+\cdots =\frac{\pi^2}{6}
\end{equation}

\begin{equation*} % adding * here will hide the equation number
\sum_{k=1}^{n} k^3=\frac{n^2(n+1)^2}{4}
\end{equation*}

\begin{equation}
\lim_{\theta \to 0} \frac{\sin\theta}{\theta} = 1
\end{equation}


\begin{equation*}
\lim_{x\to \infty} \left(\frac{\sin x}{x}\right)^2 = 0  % you can use (...)^2 here, but
% \left(...\right) can automatically adjust the bracket size based on the content within for a better look.
\end{equation*}

\begin{equation}
x_{n+1} = x_n-\frac{f(x_n)}{f'(x_n)}
\end{equation}



\begin{equation}
\lnot (p \lor q) = \lnot p \land \lnot q  \nonumber % \nonumber can also hide the equation number
\end{equation}

\begin{equation}
y=e^x\Leftrightarrow x=\ln y
\end{equation}

%%%%%%%%%%%%%%%%%%%%%%%%%%%%%%%%%%%%%
\subsection{eqnarray environment}


\begin{eqnarray} % aligning multiple lines at the place specified by &..&
x + 2y -z &=&0\\
2x-3y+5z &=&3\\
-3y+2z &=&-8
\end{eqnarray}

\begin{eqnarray*}
\sin(\alpha\pm\beta) &=& \sin\alpha\cos\beta \pm \cos\alpha\sin\beta\\
\cos(\alpha\pm\beta) &=& \cos\alpha\cos\beta \mp \sin\alpha\sin\beta
\end{eqnarray*}

\begin{eqnarray}
\sin^2 \theta &=& \frac{1}{2}(1-\cos 2\theta)\\
\cos^2\theta &=& \frac{1}{2}(1+\cos 2\theta)
\end{eqnarray}

\subsection{align environment}

\begin{align} % aligning multiple lines at the place specified by &...
x + 2y -z &=0\\
2x-3y+5z &=3 \nonumber \\
-3y+2z &=-8
\end{align}

\begin{align*}
p(x)&=(x-2) \cdot (x^2-9) \\
&=(x-2)(x-3)(x+3) 
\end{align*}

\begin{align*}
(x+2)^3 &=\binom{3}{0}x^3+\binom{3}{1}x^2\cdot 2^1+\binom{3}{2}x^1\cdot 2^2+\binom{3}{3} 2^3\\[2mm] % [2mm] after \\ can add more vertical spaces between the lines
&= x^3+6x^2+12x+8
\end{align*}

%%%%%%%%%%%%%%%%%%%%%%%%%%%%%%%%%%%%%
\subsection{matrix environment}

\begin{equation}
\begin{bmatrix} % a matrix in brackets [...]
1 & 2 & 3\\
4 & 5 & 6
\end{bmatrix}
=
\begin{pmatrix} % a matrix in parentheses (...)
1 & 2 & 3\\
4 & 5 & 6
\end{pmatrix}
\end{equation}

$$
\det 
\begin{pmatrix}
1 & 2\\
3 &4
\end{pmatrix}
=
\begin{vmatrix} % a matrix in vertical lines |...|
1 & 2\\
3 &4
\end{vmatrix}
=1\cdot 4-2\cdot 3=-2
$$

\begin{eqnarray}
\begin{pmatrix}
1 & 2\\
3 &4
\end{pmatrix}^{-1}
&=&
\frac{1}{-2}\cdot
\begin{pmatrix}
4 & -2\\
-3 &1
\end{pmatrix} \nonumber \\[2mm]
&=&
\begin{pmatrix}
-2 & 1\\
1.5 &-0.5
\end{pmatrix}
\end{eqnarray}


%%%%%%%%%%%%%%%%%%%%%%%%%%%%%%%%%%%%%
\subsection{array environment}
% this environment can typeset similar matrix strucutres with more flexibilities. you can compare the examples to those above

\begin{equation*}
\left[ % specify the types of bracket around the matrix elements
\begin{array}{ccc} % specify column alignment left/center/right
1 & 2 & 3\\
4 & 5 & 6
\end{array}
\right] % specify the types of bracket around the matrix elements
=
\left( % specify the types of bracket around the matrix elements
\begin{array}{ccc}
1 & 2 & 3\\
4 & 5 & 6
\end{array}
\right) % specify the types of bracket around the matrix elements
\end{equation*}


\begin{equation}
\det 
\left(
\begin{array}{cc}
1 & 2\\
3 &4
\end{array}
\right)
=
\left|
\begin{array}{cc}
1 & 2\\
3 &4
\end{array}
\right|
=1\cdot 4-2\cdot 3=-2
\end{equation}

\begin{eqnarray}
\left(\begin{array}{cc}
1 & 2\\
3 &4
\end{array}\right)^{-1}
&=&
\frac{1}{-2}\cdot
\left(
\begin{array}{rr}
4 & -2\\
-3 &1
\end{array}\right) \\[2mm]
&=&
\left(
\begin{array}{rr}
-2 & 1\\
1.5 &-0.5
\end{array}\right) \nonumber
\end{eqnarray}


% below is another application: typesetting piecewise function
\begin{equation}
|x|= 
\left\{ % specify the left curly bracket around the content
\begin{array}{rc}
x,& x\ge 0\\
-x, & x<0
\end{array}
\right. % there is no right curly bracket needed, put . here to form a left-right pair.
\end{equation}








\end{document}