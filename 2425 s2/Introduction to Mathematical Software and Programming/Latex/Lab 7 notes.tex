\documentclass[12pt,a4paper]{article} % what's the meaning of the settings in the []? find it in your self-study links.
\usepackage{graphicx} % this package is needed for inserting image
\usepackage[margin=3cm,top=2cm,bottom=2cm]{geometry} % this package can be used for adjusting the page margins

\title{Lab 7 Notes}
\author{XXX}
\date{31/03/2025}

\begin{document}
\maketitle
\tableofcontents

\newpage

\section{Table}
\subsection{simple table}
Example 1:\\

\begin{tabular}{|c|c|}
\hline
 True&False\\
\hline
logical 1 & logical 0\\
\hline
\end{tabular}

\subsection{table environment (floating)}

Example 2:\\
\begin{table}[!htb] % h: here. t: top. b: bottom. Check self-study links for more detailed explanation.
\centering
\begin{tabular}{|l|r|} % l: left aligned, r: right aligned, c: center aligned
\hline
 True&False\\
\hline
logical 1 & logical 0\\
\hline
\end{tabular}
\caption{Simple Table}
\end{table}

\noindent Pay attention to the alignment (left/center/right) here.\\ % \noindent is to cancell the auto-indentation at the beginning of each paragraph

\noindent Example 3:\\
\begin{table}[!htbp]
\centering
\begin{tabular}{c|c|c|c}
$p$ & $q$ & $p \land q$ &  $p \lor q$\\ % math-related cells should be put into inline math mode each
\hline
true & true & true & true\\
\hline
true & false & false & true\\
\end{tabular}
\caption{Truth Table}
\end{table}

\section{Figure}

Extra package is needed for inserting images to \LaTeX\,files. The command 
\verb|\usepackage{graphicx}| must be declared in the preamble.

\subsection{simple image}
%Example 4:\\
%Here is one image captured from the website https://www.latex-project.org/\\
%\includegraphics[scale=0.5]{image 1}\\ % the image file must be in the same folder as the source code; otherwise, you need to specify the path of its location (see self-study links)
%Use ``Windows+Shift+S'' to capture the screen.

%\subsection{figure environment (floating)} \label{plot} % this is for cross-reference
%Example 5:\\
%Plot the graphs of two functions on interval $[-\pi/2,\pi/2]$ in MATLAB:[
%f(x) = \sin(x)\cos(x), \quad g(x)=1-2x^2 
%\]
%Use different colors and line styles.
%\begin{figure}[!h]
%\centering
%\includegraphics[width=0.8\textwidth]{image 2} % common ways of adjusting the figure size
%\caption{MATLAB plot}
%\end{figure}

\section{Code}
\subsection{simple code}
Example 6:\\
$y=\sqrt[3]{x}$ can be typeset by \verb|$y=\sqrt[3]{x}$|.

\subsection{verbatim environment}
Example 7:\\
The MATLAB source code for generating the figure in Section \ref{plot} is listed below. % this is for cross-reference
\begin{verbatim}
x = linspace(-pi/2,pi/2);
f = sin(x).*cos(x);
g = 1-2*x.^2;
plot(x,f,'r:',x,g,'b--','LineWidth',2)
grid on
legend('f(x)','g(x)')
title('two functions')
\end{verbatim}

\section{Font}

\subsection{font size}
Different font sizes supported in \LaTeX:\\

\noindent Example 8:\\
Here is a {\tiny sample word}.\\
Here is a {\scriptsize sample word}.\\
Here is a {\footnotesize sample word}.\\
Here is a {\small sample word}.\\
Here is a sample word in normal size.\\
Here is a {\large sample word}.\\
Here is a {\Large sample word}.\\
Here is a {\LARGE sample word}.\\
Here is a {\huge sample word}.\\
Here is a {\Huge sample word}.\\

\noindent The source code is listed below for your reference.
\begin{verbatim}
Here is a {\tiny sample word}.\\
Here is a {\scriptsize sample word}.\\
Here is a {\footnotesize sample word}.\\
Here is a {\small sample word}.\\
Here is a sample word in normal size.\\
Here is a {\large sample word}.\\
Here is a {\Large sample word}.\\
Here is a {\LARGE sample word}.\\
Here is a {\huge sample word}.\\
Here is a {\Huge sample word}.\\
\end{verbatim}

\subsection{font style}
Common font styles used in \LaTeX:\\

\noindent Example 9:\\
normal text\\
\textbf{boldface text}\\
\textit{italics text}\\
\textsc{smallcaps text}\\
\texttt{teletype text}\\
\underline{underline text}\\

\noindent The source code is listed below for your reference.
\begin{verbatim}
normal text\\
\textbf{boldface text}\\
\textit{italics text}\\
\textsc{smallcaps text}\\
\texttt{teletype text}\\
\underline{underline text}\\
\end{verbatim}

\noindent Example 10:\\
We can \textbf{\small combine} the \textbf{\large font size} and \underline{\textit{font style}} for highlighting the content, but \textsc{do not} \texttt{\huge abuse} them.


















\end{document}